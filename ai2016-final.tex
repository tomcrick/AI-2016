% RECOMMENDED %%%%%%%%%%%%%%%%%%%%%%%%%%%%%%%%%%%%%%%%%%%%%%%%%%%
\documentclass[graybox]{svmult}

% choose options for [] as required from the list
% in the Reference Guide

\usepackage{mathptmx}       % selects Times Roman as basic font
\usepackage{helvet}         % selects Helvetica as sans-serif font
\usepackage{courier}        % selects Courier as typewriter font
\usepackage{type1cm}        % activate if the above 3 fonts are
                            % not available on your system
%
\usepackage{makeidx}         % allows index generation
\usepackage{graphicx}        % standard LaTeX graphics tool
                             % when including figure files
\usepackage{multicol}        % used for the two-column index
\usepackage[bottom]{footmisc}% places footnotes at page bottom
\usepackage{tabularx}
\usepackage{booktabs}
\usepackage{pdflscape}
\usepackage{url}
\usepackage{float}
\urlstyle{same}

%\usepackage[numbers,sort]{natbib} % "sort" = "orders multiple citations into the sequence in which they 
% appear in the list of references;"
%\usepackage[style=numeric,sorting=none,natbib=true,backend=biber]{biblatex}

% see the list of further useful packages
% in the Reference Guide


\makeindex             % used for the subject index
                       % please use the style svind.ist with
                       % your makeindex program


% fix table columns
\newcolumntype{Y}{>{\centering\arraybackslash}X}

%%%%%%%%%%%%%%%%%%%%%%%%%%%%%%%%%%%%%%%%%%%%%%%%%%%%%%%%%%%%%%%%%%%%%%%%%%%%%%%%%%%%%%%%%

\begin{document}

\title*{Incorporating Emotion and Personality-Based Analysis in User-Centered Modelling}
% Use \titlerunning{Short Title} for an abbreviated version of
% your contribution title if the original one is too long
\author{Mohamed Mostafa\thanks{The original extended version of this
    paper is available here: \protect\url{http://arxiv.org/abs/1608.03061}} \and Tom Crick \and Ana C. Calderon \and Giles Oatley}
% Use \authorrunning{Short Title} for an abbreviated version of
% your contribution title if the original one is too long
\institute{Mohamed Mostafa \and Tom Crick \and Ana C. Calderon \at
  Department of Computing \& Information Systems, Cardiff Metropolitan
  University, Cardiff, UK;
  \email{{momostafa,tcrick,acalderon}@cardiffmet.ac.uk}
\and
Giles Oatley \at School of Engineering \& Information Technology,
Murdoch University, Australia;\\\email{g.oatley@murdoch.edu.au}}
%
% Use the package "url.sty" to avoid
% problems with special characters
% used in your e-mail or web address
%
\maketitle

\abstract{Understanding complex user behaviour under various
conditions, scenarios and journeys is fundamental to improving the
user-experience for a given system.  Predictive models of user
reactions, responses -- and in particular, emotions -- can aid in the
design of more intuitive and usable systems. Building on this theme,
the preliminary research presented in this paper correlates events and
interactions in an online social network against user behaviour,
focusing on personality traits. Emotional context and tone is analysed
and modelled based on varying types of sentiments that users express
in their language using the IBM Watson Developer Cloud tools. The data
collected in this study thus provides further evidence towards
supporting the hypothesis that analysing and modelling emotions,
sentiments and personality traits provides valuable insight into
improving the user experience of complex social computer systems.}

% Keywords: emotions; human computer interactions; social media;
% artificial intelligence; social analysis; affective computing.


\section{Introduction}\label{intro}

As computer systems and applications have become more widespread and
complex, with increasing demands and expectations of ever-more
intuitive human-computer interactions, research in modelling,
understanding and predicting user behaviour demands has become a
priority across a number of domains.  In these application domains, it
is useful to obtain knowledge about user profiles or models of
software applications, including intelligent agents, adaptive systems,
intelligent tutoring systems, recommender systems, e-commerce
applications and knowledge management
systems~\cite{schiaffino+amandi:2009}. Furthermore, understanding user
behaviour during system events leads to a better informed predictive
model capability, allowing the construction of more intuitive
interfaces and an improved user experience. This work can be applied
across a range of socio-technical systems, impacting upon both
personal and business computing.

%,oatley-et-al_dasc2015,blamey-et-al-2012,blamey-et-al-2013}. 
%,oatley-et-al-soccogcomp2015}. Big social data offers the potential for new insights into human behaviour
We are particularly interested in the relationship between digital
footprint and behaviour and
personality~\cite{oatley+crick:2014,oatley-et-al_dasc2015}. A wide
range of pervasive and often publicly available datasets encompassing
digital footprints, such as social media activity, can be used to
infer
personality~\cite{lambiotte+kosinski:2014,oatley-et-al-soccogcomp2015}
and development of robust models capable of describing individuals and
societies~\cite{lazer-et-al:2009}. Social media has been used in
varying computer system approaches; in the past this has mainly been
the textual information contained in blogs, status posts and photo
comments~\cite{blamey-et-al-2012,blamey-et-al-2013}, but there is also
a wealth of information in the other ways of interacting with online
artefacts. From sharing and gathering of information and data, to
catering for marketing and business needs; it is now widely used as
technical support for computer system platforms.

The work presented in this paper is builds upon previous work in
psycholinguistic science and aims to provide further insight into how
the words and constructs we use in our daily life and online
interactions reflect our personalities and our underlying emotions. As
part of this active research field, it is widely accepted that written
text reflects more than the words and syntactic constructs, but also
conveys emotion and personality traits~\cite{pennebaker+king:1999}. As
part of our work, the IBM Watson Tone Analyzer (part of the IBM Watson
Developer Cloud toolchain) has been used to identify emotion tones in
the textual interactions in an online system, building on previous
work in this area that shows a strong correlation between the word
choice and personality, emotions, attitude and cognitive processes,
providing further evidence that it is possible to profile and
potentially predict users’ identity~\cite{fast+funder:2008}. The
{\emph{Linguistic Inquiry and Word Count}} (LIWC) psycholinguistics
dictionary~\cite{pennebaker-et-al:2001,tausczik+pennebaker:2010} is
used to find psychologically meaningful word categories from word
usage in writing; the work presented here provides a modelling and
analysis framework, as well as associated toolchain, for further
application to larger datasets to support the research goal of
improving user-centered modelling.

The rest of the paper is structured as follows: in
Sections~\ref{dataanalysis} and \ref{keymodels} we present our data,
the statistical analysis and identify the key elements of our model;
in Section~\ref{conclusions} we summarise the main contributions of
this paper, as well as making clear recommendations for future
research.

%\section{Personality Insight}\label{personality}

%Numerous studies have suggested key words and phrases can signal underlying tendencies and that this can form the basis of identifying certain aspects of personality~\citep{iacobelli-et-al:2011,pennebaker+king:1999,oberlander+gill:2004,oberlander+gill:2006}. \citet{scherer:1984} introduced a valuable classification with the following distinctions between emotions, moods, interpersonal stances, attitudes and personality traits. By observing the occurrences of words that relate to these five categories, we can infer aspects of the holder's psychological state. For instance, sentiment analysis or opinion mining (utilising open-source software such as{\emph{SentiWordNet}}); as well as the use of features from psycholinguistic databases such as {\emph{LIWC}}~\citep{pennebaker-et-al:2001} can be used to create a range of statistical models for each of the Five Factor personality traits~\citep{mairesse-et-al:2007}. 

%The ``Big Five'' model, focuses on five dimensions, namely: {\emph{Agreeableness}}, {\emph{Conscientiousness}}, {\emph{Extraversion}}, {\emph{Neuroticism}} and {\emph{Openness}}~\citep{norman:1963,peabody+goldberg:1989}. It should be noted that while researchers have continued to work with the Five Factors model, there are well known limitations~\cite{eysenck:1992,paunonen+jackson:2000,block:2010} that are often overlooked; however, over the past fifty years the Five Factor model has become a standard in psychology~\cite{mairesse-et-al:2007}, developing a large corpus of work to compare against.

% The IBM Watson Tone
%Analyzer\footnote{\url{http://www.ibm.com/watson/developercloud/tone-analyzer.html}} is a cloud-based framework to infer emotions from a given text; it uses linguistic analysis to detect three types of tones from written text: emotions, social tendencies, and writing style. Emotions identified include {\emph{Anger}}, {\emph{Fear}}, {\emph{Joy}},{\emph{Sadness}} and {\emph{Disgust}}; identified social tendencies include the Big Five personality traits (as described above); identified writing styles include {\emph{Confident}}, {\emph{Analytical}} and {\emph{Tentative}}. To derive emotion scores from text, IBM Watson Tone Analyzer uses a stacked generalisation-based ensemble framework using machine learning techniques to achieve greater predictive  accuracy~\citep{costa+mccrae:1992}. LIWC is used to find psychologically meaningful word categories from word usage in writing.

% Features such as n-grams (unigrams, bigrams and trigrams),
% punctuation, emoticons, curse words, greeting words (such as
% ``{\emph{hello}}'', ``{\emph{hi}}'' and ``{\emph{thanks}}'') and
% sentiment polarity are fed into machine learning algorithms to
% classify emotion categories~\citep{fellbaum:2006}.

%IBM Watson Personality Insights\footnote{\url{http://www.ibm.com/watson/developercloud personality-insights.html}} provides a deeper understanding of people's personality characteristics, needs, and values to drive personalisation. It extracts and analyses a spectrum of personality attributes to help discover actionable insights about people and entities; the service outputs personality characteristics that are divided into three  dimensions: the Big Five, {\emph{Values}} and {\emph{Needs}}. While some services are contextually specific depending on the domain model and content, Personality Insights usually only requires a minimum of 3500+ words of any text.


\section{Data Analysis \& Feature Extraction}\label{dataanalysis}

%\subsection{Overview of the Data}
Our dataset comes from an online portal for a European Union
international scholarship mobility hosted at a UK university. The
dataset was generated from interactions between users and a complex
online information system, namely the online portal for submitting
applications. The whole dataset consists of users (N=391),
interactions and comments (N=1390) as responses to system status and
reporting their experience with using the system. Google Analytics has
been used to track user behaviour and web statistics (such as
impressions); this data from has been used to identify the server's
status and categorised the status as two stages: {\emph{Idle}, where
the system had a higher number of active sessions; and marked as
{\emph{Failure}}, where the system had a lower number of sessions
engaged. Interactions were first grouped by server status, then sent
to the IBM Watson Tone Analyzer to generate emotion social tone
scores. In what follows {\emph{Failure}} status shows a significant
difference in overall {\emph{Anger}} in different status; furthermore,
the {\emph{Joy}} parameter shows a significant difference with the
system in {\emph{Idle}} and {\emph{Failure}} status. However
{\emph{Fear}} and {\emph{Sadness}} parameters is almost the same, even
with the system in {\emph{Idle}} status.

% Figure~\ref{fig:googleanalytics} provides
% a plot of web traffic from Google Analytics over a specific day, clearly
% showing the drop at 20:00 where the system had been identified as in
% the {\emph{Failure}} state.

% \begin{figure}[!ht]
% \centering
% \includegraphics[width=\columnwidth]{images/googleanalytics}
% \caption{Google Analytics profile shows behaviour of the system over a
%   24 hour period (timeline during the day vs. number of active sessions)}
% \label{fig:googleanalytics} 
% \end{figure}

% removed footnote:
% \footnote{\url{http://www.ibm.com/watson/developercloud/tone-analyzer.html}}



% Features such as n-grams (unigrams, bigrams and trigrams),
% punctuation, emoticons, curse words, greeting words (such as
% ``{\emph{hello}}'', ``{\emph{hi}}'' and ``{\emph{thanks}}'') and
% sentiment polarity are fed into machine learning algorithms to
% classify emotion categories~\citep{fellbaum:2006}.
% \begin{figure}[!ht]
% \centering
% \includegraphics[width=\columnwidth]{images/emotiontone}
% \caption{Overall emotion tone response to server failure/idle status}
% \label{fig:emotiontone} 
% \end{figure}

% removed footnote:
% \footnote{\url{http://www.ibm.com/watson/developercloud personality-insights.html}}
We identified the user's personality based on analysis of their
Facebook interactions, namely by collecting all comments from the
users, again using the IBM Watson Personality Insights tool. However,
a number of users in the dataset had completed the Big Five
questionnaire (N=44); for these users, their Big Five scores have
been used instead. The second stage involved grouping the comments
based on server status and segmenting these interactions by user; this
allowed us to investigate the impact of server status in the emotion
of the user and investigate the Big Five dimension as a constant
parameter. By investigating the relationship between personality trait
dimensions and the social emotion tones, we are able to find the
highest correlation to identify the key elements of the potential
model by applying linear regression and Pearson correlation. This
allowed building of a neural network multilayer perception using the
potential key elements with higher correlations.

The data collected from the social media interactions was grouped by
users and via IBM Watson Personality Insights, we were able to
identify the Big Five personality traits for each user. Using the IBM
Watson Tone Analyzer, the data was grouped by user's comments and
server status ({\emph{Failure}}, {\emph{Idle}}) to identify the social
emotion tone for each user.

% Table~\ref{tab:sample}
% shows a sample of data used in this analysis, with each row
% representing a unique user, and each column represents the Big Five
% traits, social emotion tones and server status.

% \begin{table}[!ht]
% \centering
% \caption{Example data snapshot used in the analysis}
% \resizebox{\columnwidth}{!}{%
% \begin{tabular}{@{}rrrrrrrrrrr@{}}
% \toprule
% \multicolumn{1}{c}{Openness} & \multicolumn{1}{c}{Conscientiousness} & \multicolumn{1}{c}{Extraversion} & \multicolumn{1}{c}{Agreeableness} & \multicolumn{1}{c}{Neuroticism} & \multicolumn{1}{c}{anger} & \multicolumn{1}{c}{disgust} & \multicolumn{1}{c}{fear} & \multicolumn{1}{c}{joy} & \multicolumn{1}{c}{sadness} & \multicolumn{1}{c}{Server  Status} \\ 
% \midrule
% 0.528 & 0.523 & 0.537 & 0.653 & 0.511 & 0.217821 & 0.793375 & 0.501131 & 0.031477 & 0.284936 & Failure \\
% 0.252 & 0.063 & 0.037 & 0.266 & 0.989 & 0.542857 & 0.084615 & 0.178302 & 0.224453 & 0.264283 & Failure \\
% 0.817 & 0.571 & 0.157 & 0.012 & 0.401 & 0.162798 & 0.166694 & 0.213870 & 0.410916 & 0.220049 & Failure \\
% 0.197 & 0.130 & 0.180 & 0.419 & 0.990 & 0.468938 & 0.259794 & 0.350803 & 0.037265 & 0.636412 & Failure \\
% 0.155 & 0.079 & 0.081 & 0.226 & 0.975 & 0.539162 & 0.219993 & 0.431932 & 0.011625 & 0.642158 & Failure \\
% 0.158 & 0.281 & 0.332 & 0.510 & 0.869 & 0.419015 & 0.162022 & 0.213941 & 0.066892 & 0.686369 & Failure \\
% 0.817 & 0.571 & 0.157 & 0.012 & 0.401 & 0.041602 & 0.026298 & 0.141606 & 0.651962 & 0.106500 & Failure \\
% 0.058 & 0.038 & 0.147 & 0.375 & 0.989 & 0.449222 & 0.057946 & 0.181654 & 0.158412 & 0.547968 & Idle \\
% 0.178 & 0.138 & 0.800 & 0.564 & 0.828 & 0.207497 & 0.096643 & 0.093218 & 0.769316 & 0.162241 & Idle \\
% 0.105 & 0.463 & 0.792 & 0.704 & 0.041 & 0.134487 & 0.257145 & 0.195858 & 0.181699 & 0.509379 & Idle \\
% 0.589 & 0.479 & 0.147 & 0.339 & 0.828 & 0.360527 & 0.240875 & 0.321188 & 0.117492 & 0.212762 & Idle \\
% 0.338 & 0.235 & 0.104 & 0.304 & 0.869 & 0.164107 & 0.015058 & 0.230148 & 0.629562 & 0.356028 & Idle \\
% 0.204 & 0.203 & 0.480 & 0.329 & 0.892 & 0.625891 & 0.193692 & 0.242459 & 0.153679 & 0.166561 & Idle \\
% 0.689 & 0.968 & 0.805 & 0.465 & 0.029 & 0.246246 & 0.080353 & 0.123761 & 0.807537 & 0.135646 & Idle \\
% 0.093 & 0.175 & 0.642 & 0.563 & 0.875 & 0.279503 & 0.045658 & 0.207278 & 0.088724 & 0.505607 & Idle \\
% 0.277 & 0.296 & 0.276 & 0.332 & 0.892 & 0.499199 & 0.143897 & 0.269725 & 0.188664 & 0.285462 & Idle \\
% 0.055 & 0.095 & 0.783 & 0.699 & 0.935 & 0.450997 & 0.153940 & 0.263070 & 0.350778 & 0.116282 & Idle \\ 
% \bottomrule
% \end{tabular}%
% }
% \label{tab:sample}
% \end{table}

\section{Statistical Analysis and Key Elements of Model}\label{keymodels}

As part of modelling the users' responses and behaviour, in order to
build a conceptual framework model, we applied linear regression to
investigate the relationship between the Big Five personality
dimensions and the emotion tones.


\begin{figure}[!ht]
\centering
\includegraphics[width=0.32\columnwidth]{images/opennessplot.png}
\includegraphics[width=0.32\columnwidth]{images/extraversionplot.png}
\includegraphics[width=0.32\columnwidth]{images/conscientiousnessplot.png}
\includegraphics[width=0.32\columnwidth]{images/agreeablenessplot.png}
\includegraphics[width=0.32\columnwidth]{images/neuroticismplot.png}
\caption{Scatterplots of the Big Five dimension (dependent variables) and
  social emotion tones (independent variables)}
\label{fig:scatterplots} 
\end{figure}


\begin{table}[!ht]
\centering
\caption{Linear regression coefficients}
\resizebox{\columnwidth}{!}{%
\begin{tabular}{@{}llll|lll|lll|lll|lll@{}}
%\begin{tabularx}{\textwidth}{llll|lll|lll|lll|lll}
\toprule
& \multicolumn{3}{c}{Openness} & \multicolumn{3}{c}{Extraversion} &
\multicolumn{3}{c}{Conscientiousness} &
\multicolumn{3}{c}{Agreeableness} & \multicolumn{3}{c}{Neuroticism} \\

\midrule
           
& \multicolumn{1}{c}{B} & \multicolumn{1}{c}{t} &
\multicolumn{1}{c}{Sig} & \multicolumn{1}{c}{B} &
\multicolumn{1}{c}{t} & \multicolumn{1}{c}{Sig} &
\multicolumn{1}{c}{B} & \multicolumn{1}{c}{t} &
\multicolumn{1}{c}{Sig} & \multicolumn{1}{c}{B} &
\multicolumn{1}{c}{t} & \multicolumn{1}{c}{Sig} &
\multicolumn{1}{c}{B} & \multicolumn{1}{c}{t} &
\multicolumn{1}{c}{Sig} \\


(constant) & 0.356                 & 3.282                 & 0.001                   & 0.162                 & 1.642                 & 0.101                   & 0.16                  & 1.623                 & 0.105                   & 0.297      & 2.831     & 0.005    & 0.828          & 9.934   & 0      \\
anger      & -0.063                & -0.735                & 0.463                   & 0.064                 & 0.831                 & 0.406                   & 0.124                 & 1.592                 & 0.112                   & 0.024      & 0.293     & 0.769    & 0.116          & 1.767    & 0.078     \\
disgust    & 0.478                 & 4.354                 & 0                       & 0.114                 & 1.142                 & 0.253                   & 0.255                 & 2.551                 & 0.011                   & -0.061     & -0.574    & 0.566    & -0.363         & -4.303    & 0    \\
fear       & 0.065                 & 0.534                 & 0.594                   & 0.172                 & 1.549                 & 0.122                   & 0.04                  & 0.356                 & 0.722                   & 0.093      & 0.783     & 0.434    & -0.023         & -0.241    & 0.81    \\
joy        & 0.066                 & 0.561                 & 0.575                   & 0.446                 & 4.179                 & 0                       & 0.436                 & 4.058                 & 0                       & 0.188      & 1.652     & 0.099    & -0.487         & -5.39   & 0      \\
sadness    & -0.226                & -2.118                & 0.035                   & -0.185                & -1.906                & 0.057                   & -0.03                 & -0.313                & 0.754                   & 0.014      & 0.132     & 0.895    & 0.233          & 2.841     & 0.005    \\

\bottomrule

%\end{tabularx} 
\end{tabular}%
}
\label{tbl:linreg}
\end{table}

Linear regressions (presented in Table~\ref{tbl:linreg} and
Figure~\ref{fig:scatterplots}) do not show significant correlations
between the Big Five dimensions and the social emotion tones. There
are, however correlations that can be used as key elements for the
model; namely the correlation of {\emph{Openness} and {\emph{Disgust}}
(0.479), {\emph{Extraversion}} and {\emph{Joy}} (0.446 with p-value of
0), {\emph{Conscientiousness}} and {\emph{Joy}} (0.436),and
\emph{Disgust}} with 0.255.{\emph{Agreeableness}}, does not appear to
have a high impact in the social emotion parameters, with the highest
correlation being 0.188 with {\emph{Joy}}, which can be overlooked as
a useful factor in the model. {\emph{Neuroticism}} and {\emph{Disgust}
is -0.363, {\emph{Joy} is -0.487 and p-value is zero is both cases;
and {\emph{Sadness}} with 0.233. All correlation values are
$<$0.5. However, {\emph{Agreeableness}} does not have a linear
relationship with any of the social emotion tones. Furthermore, the
social emotion tones that have a potential linear relationship are
{\emph{Disgust}}, {\emph{Joy}} and {\emph{Sadness}}, since the three
tones have a correlation between $>$0.3 and $<$0.5.

Previous linear regression analysis suggested that the following Big
Five dimensions: {\emph{Openness}}, {\emph{Extraversion}},
{\emph{Conscientiousness}} and {\emph{Neuroticism}} have the highest
correlation with the social emotion tones: {\emph{Joy}},
{\emph{Sadness}} and {\emph{Disgust}}. For further analysis, the
Pearson correlation for the same dataset has been performed to compare
the output with the linear regression correlations. As noted in
Table~\ref{tab:pearson}, there is no significant correlation in both;
however, in the Pearson correlation, {\emph{Neuroticism}} has the
highest correlation values across emotion tones, especially
{\emph{Anger}}, {\emph{Joy}} and {\emph{Sadness}}. {\emph{Joy}} does
have a correlation with all Big Five dimensions except for
{\emph{Agreeableness}} which agrees with the previous
analysis. However, {\emph{Disgust}} does not have a strong correlation
with any of the Big Five dimensions, which deviates from the previous
analysis.

\begin{table}[!ht]
\centering
\caption{Pearson correlations}
\begin{tabular}{@{}rrrrrr@{}}
\toprule
                  & \multicolumn{1}{c}{Anger}  & \multicolumn{1}{c}{Disgust} & \multicolumn{1}{c}{Fear}   & \multicolumn{1}{c}{Joy}    & \multicolumn{1}{c}{Sadness} \\ 
\midrule
Openness          & -0.098 & 0.231   & 0.043  & 0.035  & -0.151  \\
Conscientiousness & -0.111 & -0.001  & -0.113 & 0.267  & -0.19   \\
Extraversion      & -0.175 & -0.077  & -0.071 & 0.349  & -0.291  \\
Agreeableness     & -0.068 & -0.089  & -0.027 & 0.14   & -0.069  \\
Neuroticism       & 0.375  & -0.037  & 0.153  & -0.488 & 0.379   \\ 

\bottomrule
\end{tabular}
\label{tab:pearson}
\end{table}


%\section{Key Elements of the Model}\label{model}

According to the output of the statistical analysis presented in
Table~\ref{tbl:linreg} (linear regression) and Table~\ref{tab:pearson}
(Pearson correlation), the Big Five dimension identified as the key
elements from the personality traits are: {\emph{Openness}},
{\emph{Extraversion}}, {\emph{Conscientiousness}} and
{\emph{Neuroticism}}. The statistical analysis agrees that
{\emph{Agreeableness}} does not have a significant correlation across
any of the social emotion tones. The social emotion tones to be used
as key input elements for the proposed model are {\emph{Joy}},
{\emph{Sadness}}, {\emph{Anger}} and {\emph{Disgust}}; although the
{\emph{Anger}} tone did not show any significant correlation in linear
regression analysis, the value of the Pearson correlation coefficient
is between 0.3 and 0.5 which can be used as input for the model.

% \begin{table}[!ht]
% \centering
% \caption{Re-evaluation output of proposed model}
% \begin{tabular}{lrr}
% Correctly classified instances:   & 43     & ({\emph{75.44\%}}) \\
% Incorrectly classified instances: & 14     & ({\emph{24.56\%}}) \\
% Kappa statistic:                  & 0.5295 &         \\
% Mean absolute error:              & 0.3432 &         \\
% Root mean squared error:          & 0.4246 &         \\
% Total number of instances:        & 57     &        
% \end{tabular}
% \label{tab:reeval}
% \end{table}

The dataset used to build this model is based upon a number of users
(N=391), eight inputs ({\emph{Openness}}, {\emph{Extraversion}},
{\emph{Conscientiousness}}, {\emph{Neuroticism}}, {\emph{Joy}},
{\emph{Sadness}}, {\emph{Anger}} and {\emph{Disgust}}) and the
class/output variable as the server status (where No: System
{\emph{Failure}} and Yes: System {\emph{Idle}}). The total number of
the instances for the testing set is 57; the output of the model shows
a 75.44\% corrected predicted instances and 24.56\% incorrectly
classified instances (kappa statistic: 0.5295; mean absolute error:
0.3432; RMS error: 0.4246). As this has been performed
on a small subset of the overall larger project dataset, the output
data is encouraging and provides the infrastructure for further
analysis and research to exploit the full dataset.

\section{Conclusions and Future Work}\label{conclusions}


%Ana: I think we perhaps leave this out, as Mo's research should stand
%on its own merit, rather than linking it to all the work Giles is
%doing/has done. It's more concerned with social media I think,so the
%second paragraph seems more useful, and we can always include this in
%a full version.
% TC: agreed, but useful for context, plus links to Mo's other paper.

In this paper, preliminary results from a larger ongoing theme of
research to profile online/digital behaviour have been
presented\footnote{See original extended paper:
\url{http://arxiv.org/abs/1608.03061}}; the objective is to build a
conceptual framework to improve user experience and computer system
architecture design. The research also concerns applicability in
interested in profiling complex behaviours and psychopathies using
social network analysis, particularly for crime
informatics~\cite{oatley+crick:2015}. Previous work in this space
analysed the document uploading behaviour (such as motivation letters,
and social media interactions) of the applicants of the international
scholarship mobility portal; by examining the upload footprint for the
users we were able to determine several classes of
behaviour~\cite{oatley-et-al-soccogcomp2015}.

Social media is used, not only as a content and sharing platform, but
also as a platform for technical support for various of online
applications and services. This paper demonstrates the analysis of one
such online application that has used Facebook as technical support
platform for the users. We have produced a model that can predict
server status based on personality traits and social emotion tones, by
investigating the linear regression and Pearson correlation to
identify the key elements to be used as input for the neural network
to build this model ({\emph{Openness}}, {\emph{Extraversion}},
{\emph{Conscientiousness}}, {\emph{Neuroticism}}, {\emph{Joy}},
{\emph{Sadness}}, {\emph{Anger}} and {\emph{Disgust}}). The produced
model shows a good potential start for further data analysis, with
75\% accuracy in predication based on 57 test cases. Furthermore the
available of high-quality, low-cost and adaptable tools provided by
the IBM Watson Developer Cloud, provide significant further
opportunities to integrate linguistic analysis into this research
domain. The model produced from this work provides a number of
recommendations for future research to further incorporate emotion and
personality-based analysis in user-centered modelling. In particular,
expanding the dataset by gathering more data from similar types of
interactions, as well as technical queries; annotating and
categorising the dataset by gender to investigate the relationship
between gender and emotion raised by the user in different computer
system statuses; as well as exploring different computer events not
only limited to {\emph{Idle}} and {\emph{Failure}}, but including more
complex events e.g. account hacked, system speed, unexpected error and
unsaved data.

% (e.g. IBM Watson Tone Analyzer and IBM
% Watson Personality Insights tools)

% \begin{acknowledgement}
% If you want to include acknowledgments of assistance and the like at the end of an individual chapter please use the \verb|acknowledgement| environment -- it will automatically render Springer's preferred layout.
% \end{acknowledgement}

% \section*{Appendix}
% \addcontentsline{toc}{section}{Appendix}

%\begin{landscape}

%\end{landscape}


% bib
\bibliographystyle{spmpsci}
\bibliography{ai2016}

\end{document}
