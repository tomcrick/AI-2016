% RECOMMENDED %%%%%%%%%%%%%%%%%%%%%%%%%%%%%%%%%%%%%%%%%%%%%%%%%%%
\documentclass[graybox]{svmult}

% choose options for [] as required from the list
% in the Reference Guide

\usepackage{mathptmx}       % selects Times Roman as basic font
\usepackage{helvet}         % selects Helvetica as sans-serif font
\usepackage{courier}        % selects Courier as typewriter font
\usepackage{type1cm}        % activate if the above 3 fonts are
                            % not available on your system
%
\usepackage{makeidx}         % allows index generation
\usepackage{graphicx}        % standard LaTeX graphics tool
                             % when including figure files
\usepackage{multicol}        % used for the two-column index
\usepackage[bottom]{footmisc}% places footnotes at page bottom
\usepackage{url}
\urlstyle{same}

\usepackage[numbers,sort]{natbib} % "sort" = "orders multiple citations into the sequence in which they 
% appear in the list of references;"
%\usepackage[style=numeric,sorting=none,natbib=true,backend=biber]{biblatex}

% see the list of further useful packages
% in the Reference Guide


\makeindex             % used for the subject index
                       % please use the style svind.ist with
                       % your makeindex program

%%%%%%%%%%%%%%%%%%%%%%%%%%%%%%%%%%%%%%%%%%%%%%%%%%%%%%%%%%%%%%%%%%%%%%%%%%%%%%%%%%%%%%%%%

\begin{document}

\title*{Incorporating emotion and personality-based analysis in user-centered modelling}
% Use \titlerunning{Short Title} for an abbreviated version of
% your contribution title if the original one is too long
\author{Mohamed Mostafa \and Ana C. Calderon \and Tom Crick \and Giles Oatley}
% Use \authorrunning{Short Title} for an abbreviated version of
% your contribution title if the original one is too long
\institute{Mohamed Mostafa \and Ana C. Calderon \and Tom Crick\at
  Department of Computing \& Information Systems, Cardiff Metropolitan
  University, Cardiff, UK;
  \email{{momostafa,acalderon,tcrick}@cardiffmet.ac.uk}
\and
Giles Oatley \at School of Engineering \& Information Technology,
Murdoch University, Australia;\\\email{g.oatley@murdoch.edu.au}}
%
% Use the package "url.sty" to avoid
% problems with special characters
% used in your e-mail or web address
%
\maketitle

\abstract{Understanding user behaviour under varying conditions,
scenarios and journeys is fundamental to the improvement of the
user-experience for a given system. Predictive models of user
reactions and responses can aid in the design of more intuitive and
usable systems. The research presented in this paper correlates events
and interactions in an online social network against user behaviour,
focusing on personality traits. Emotional context and tone is analysed
and modelled based on varying types of sentiments that users express
in their language using the IBM Watson Developer Cloud tools. The data
collected in this study thus provides further significant evidence
towards supporting the hypothesis that analysing and modelling
emotions, sentiments and personality traits provides valuable insight
into improving the user experience of complex social computer
systems.}

% Keywords: emotions; human computer interactions; social media;
% artificial intelligence; social analysis; affective computing.



\section{Introduction}\label{intro}

As computer system applications become more complex, with more complex
demands of ever more intuitive human-application interaction, research
in predicting and understanding user behaviour, applied to particular
systems becomes ever more important, impacting elements of daily
societal life, both professionally and personally. Understanding user
behavior, during particular events, leads to a more informed
predictive model, thus allowing the construction of more intuitive
interfaces and a better user experience. Our work, based on
psycholinguistics science, aims to understand whether the words we use
in our daily life reflect our personalities and what we
fell. Psycholinguistics is a well established and active research
field, and it widely accepted that written text can reflect more than
words, it conveys emotion and personality traits.

In this paper the IBM Watson tone analyzer have been used to identify
emotion tones in the text, the research in this area has shown a
strong correlation between the word choice and personality, emotions,
attitude and thought process. This provides further evidence that it
is possible to profile users’ identity Fast and Funder (2008). Most of
the work based on the Linguistic Inquiry and Word Count (LIWC)
psycholinguistics dictionary Tausczik \& Pennebaker, 2010, and
Pennebaker et al., 2007. The LIWC is used to find psychologically
meaningful word categories from word usage in writing.

\section{Methodology}\label{method}

\subsection{Dataset}

Social media has been used in varying computer system approaches,
varying from sharing and gathering of information and data, to
catering for marketing and business needs. Furthermore, it is also
used as technical support for computer system platforms (Thompson,
2009). 

% Fig1 here ***

According to a survey conducted by Bob Thompson (2009) more than 60\%,
agrees with the statement that social media have been used as a
technical support for posting technical issues for computer system
(Figure 1). Our data set was generated from interactions between users
and complex scholarship system for EU funds. The whole set consists of
391 users and 1390 comment posted by users as response to system
status and reporting their experience with the system.

Google analytics have been installed in the web application to track
user’s behavior and web pages’ impressions, however, the data from
Google analytics have been used to identify the server’s status and
divided the status to two stages idle, where system had higher number
of sessions and system marked as failure where system had a lower
session engaged. As shown in Figure 2, is an screen shot of google
analytic in one day and clearly shows the drop at 8 pm where the
system has been identified as failure.



% \begin{acknowledgement}
% If you want to include acknowledgments of assistance and the like at the end of an individual chapter please use the \verb|acknowledgement| environment -- it will automatically render Springer's preferred layout.
% \end{acknowledgement}

% bib
\bibliographystyle{spmpsci}
\bibliography{ai2016}

\end{document}
